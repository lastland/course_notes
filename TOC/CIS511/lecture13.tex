%
% This is the LaTeX template file for lecture notes for CS294-8,
% Computational Biology for Computer Scientists.  When preparing 
% LaTeX notes for this class, please use this template.
%
% To familiarize yourself with this template, the body contains
% some examples of its use.  Look them over.  Then you can
% run LaTeX on this file.  After you have LaTeXed this file then
% you can look over the result either by printing it out with
% dvips or using xdvi.
%
% This template is based on the template for Prof. Sinclair's CS 270.

\documentclass[twoside]{article}
\usepackage{graphics}
\usepackage{amsfonts}
\usepackage{tikz}
\usepackage{amsmath}
\usepackage{IEEEtrantools}
\usepackage{amsmath}

\mathchardef\mhyphen="2D % Define a "math hyphen"

\usetikzlibrary{chains,fit,shapes}
\setlength{\oddsidemargin}{0.25 in}
\setlength{\evensidemargin}{-0.25 in}
\setlength{\topmargin}{-0.6 in}
\setlength{\textwidth}{6.5 in}
\setlength{\textheight}{8.5 in}
\setlength{\headsep}{0.75 in}
\setlength{\parindent}{0 in}
\setlength{\parskip}{0.1 in}

%
% The following commands set up the lecnum (lecture number)
% counter and make various numbering schemes work relative
% to the lecture number.
%
\newcounter{lecnum}
\renewcommand{\thepage}{\thelecnum-\arabic{page}}
\renewcommand{\thesection}{\thelecnum.\arabic{section}}
\renewcommand{\theequation}{\thelecnum.\arabic{equation}}
\renewcommand{\thefigure}{\thelecnum.\arabic{figure}}
\renewcommand{\thetable}{\thelecnum.\arabic{table}}

%
% The following macro is used to generate the header.
%
\newcommand{\chno}[4]{
   \pagestyle{headings}
   \thispagestyle{plain}
   \newpage
   \setcounter{lecnum}{#1}
   \setcounter{page}{1}
   \noindent
   \begin{center}
   \framebox{
      \vbox{\vspace{2mm}
    \hbox to 6.28in { {\bf CIS 511: Theory of Computation
                        \hfill Feb 23, 2017} }
       \vspace{4mm}
       \hbox to 6.28in { {\Large \hfill Lecture #1: #2  \hfill} }
       \vspace{2mm}
       \hbox to 6.28in { {\it Professor #3 \hfill #4} }
      \vspace{2mm}}
   }
   \end{center}
   \markboth{Lecture #1: #2}{Lecture #1: #2}
   {\bf NB}: {\it These notes are from CIS511 at Penn. The course followed Michael Sipser's \textit{Introduction to the Theory of Computation (3ed)} text.}
   \vspace*{4mm}
}

%
% Convention for citations is authors' initials followed by the year.
% For example, to cite a paper by Leighton and Maggs you would type
% \cite{LM89}, and to cite a paper by Strassen you would type \cite{S69}.
% (To avoid bibliography problems, for now we redefine the \cite command.)
% Also commands that create a suitable format for the reference list.
\renewcommand{\cite}[1]{[#1]}
\def\beginrefs{\begin{list}%
        {[\arabic{equation}]}{\usecounter{equation}
         \setlength{\leftmargin}{2.0truecm}\setlength{\labelsep}{0.4truecm}%
         \setlength{\labelwidth}{1.6truecm}}}
\def\endrefs{\end{list}}
\def\bibentry#1{\item[\hbox{[#1]}]}

%Use this command for a figure; it puts a figure in wherever you want it.
%usage: \fig{NUMBER}{SPACE-IN-INCHES}{CAPTION}
\newcommand{\fig}[3]{
			\vspace{#2}
			\begin{center}
			Figure \thelecnum.#1:~#3
			\end{center}
	}
% Use these for theorems, lemmas, proofs, etc.
\newtheorem{theorem}{Theorem}[lecnum]
\newtheorem{lemma}[theorem]{Lemma}
\newtheorem{proposition}[theorem]{Proposition}
\newtheorem{claim}[theorem]{Claim}
\newtheorem{corollary}[theorem]{Corollary}
\newtheorem{definition}[theorem]{Definition}
\newenvironment{proof}{{\bf Proof:}}{\hfill\rule{2mm}{2mm}}

% **** IF YOU WANT TO DEFINE ADDITIONAL MACROS FOR YOURSELF, PUT THEM HERE:

\begin{document}
%FILL IN THE RIGHT INFO.
%\lecture{**LECTURE-NUMBER**}{**DATE**}{**LECTURER**}{**SCRIBE**}
\chno{13}{Wrapping up $NP$ and Beginning Space Complexity}{Sampath Kannan}{Zach Schutzman}
%\footnotetext{These notes are partially based on those of Nigel Mansell.}

% **** YOUR NOTES GO HERE:

% Some general latex examples and examples making use of the
% macros follow.  
%**** IN GENERAL, BE BRIEF. LONG SCRIBE NOTES, NO MATTER HOW WELL WRITTEN,
%**** ARE NEVER READ BY ANYBODY.


\section*{More $NP\mhyphen Complete$ness}

Other problems are $NP\mhyphen Complete$, the reduction for $3COLOR$ creates gadgets from each clause and each variable (see any Algorithms book).

There is an obvious reduction from $3COLOR$ to $4COLOR$ (and beyond).  Add a new vertex, connect it to every vertex in your original graph.

$SUBSETSUM$, the question of whether a set of numbers contains a subset that sums to a particular value $k$ is $NP\mhyphen Complete$.  The reduction is from $3SAT$.  Given a formula $\phi$ with $m$ clauses and $n$ variables, we create $2n$ numbers with $n+m$ base $7$ digits.  Each variable gets a $1$ in positions indicating the index of the variable and the clauses it appears in.  $\phi$ is satisfiable if and only if there is a subset that sums to $11\dots111$ in the first $n$ positions, and $4\dots44$ in the remaining $m$ positions, by introducing some dummy numbers which are all zeros in the first $n$ positions and $1\ or \ 2$ in the corresponding clause position.


We know that $P\subseteq NP$, $NP\mhyphen Complete \subset NP$

\definition{A language $L$ is in $Co\mhyphen NP$ if its complement is in $NP$.  Equivalently, given $x\notin L$, a verifier can check non-membership given the right certificate (easy to check that something is not a solution).}

\textbf{Example:} $TAUT = \{ \phi | \phi \ is  \ satistfied \ by \ every \ assignment\}$ is in $Co\mhyphen NP$.  If a $\phi$ is not in $L$, then any non-satisfying assignment can be quickly checked.


We don't know if $NP=Co\mhyphen NP$.

If $P=NP$, then $NP = Co\mhyphen NP$.

If $NP\neq Co\mhyphen NP$, then $P\neq NP$.


\section*{Space Complexity}

\definition{\textbf{Space complexity} refers to the number of cells of tape scanned by the head of a Turing machine in running an input.}

\definition{$DSPACE(s(n))$ is the set of languages recongized by a deterministic TM in $O(s(n))$ space.}

\definition{$NSPACE(s(n))$ is the set of languages recognized by a non-deterministic TM in $O(s(n))$ space.}

Let's look at $SAT\in NP$.  What is its space complexity?  Let's say $n$ is the length of the input and $k$ is the number of variables.  If we don't care about time, we can represent the assignment as a $k$-bit number and try one assignment at a time.  We only need $n+k+c$ (where $c$ is some small fixed workspace).  Therefore, $SAT\in DSPACE(n)$.

Let $L_{NANFA} = \{ \langle M \rangle | M \ is \ an \ NFA, \ L(M)\neq \Sigma^* \}$.  We don't know if this is in $NP$ because the naive approach for a verifier does not necessarily have a polynomial length certificate.  We can show that it is in $NSPACE(n)$.





\end{document}







