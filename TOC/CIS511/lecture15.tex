%
% This is the LaTeX template file for lecture notes for CS294-8,
% Computational Biology for Computer Scientists.  When preparing 
% LaTeX notes for this class, please use this template.
%
% To familiarize yourself with this template, the body contains
% some examples of its use.  Look them over.  Then you can
% run LaTeX on this file.  After you have LaTeXed this file then
% you can look over the result either by printing it out with
% dvips or using xdvi.
%
% This template is based on the template for Prof. Sinclair's CS 270.

\documentclass[twoside]{article}
\usepackage{graphics}
\usepackage{amsfonts}
\usepackage{tikz}
\usepackage{amsmath}
\usepackage{IEEEtrantools}
\usepackage{amsmath,amssymb,trimclip,adjustbox}

\mathchardef\mhyphen="2D % Define a "math hyphen"

\usetikzlibrary{chains,fit,shapes}
\setlength{\oddsidemargin}{0.25 in}
\setlength{\evensidemargin}{-0.25 in}
\setlength{\topmargin}{-0.6 in}
\setlength{\textwidth}{6.5 in}
\setlength{\textheight}{8.5 in}
\setlength{\headsep}{0.75 in}
\setlength{\parindent}{0 in}
\setlength{\parskip}{0.1 in}

%
% The following commands set up the lecnum (lecture number)
% counter and make various numbering schemes work relative
% to the lecture number.
%
\newcounter{lecnum}
\renewcommand{\thepage}{\thelecnum-\arabic{page}}
\renewcommand{\thesection}{\thelecnum.\arabic{section}}
\renewcommand{\theequation}{\thelecnum.\arabic{equation}}
\renewcommand{\thefigure}{\thelecnum.\arabic{figure}}
\renewcommand{\thetable}{\thelecnum.\arabic{table}}

%
% The following macro is used to generate the header.
%
\newcommand{\chno}[4]{
	\pagestyle{headings}
	\thispagestyle{plain}
	\newpage
	\setcounter{lecnum}{#1}
	\setcounter{page}{1}
	\noindent
	\begin{center}
		\framebox{
			\vbox{\vspace{2mm}
				\hbox to 6.28in { {\bf CIS 511: Theory of Computation
						\hfill Feb 30, 2017} }
				\vspace{4mm}
				\hbox to 6.28in { {\Large \hfill Lecture #1: #2  \hfill} }
				\vspace{2mm}
				\hbox to 6.28in { {\it Professor #3 \hfill #4} }
				\vspace{2mm}}
		}
	\end{center}
	\markboth{Lecture #1: #2}{Lecture #1: #2}
	{\bf NB}: {\it These notes are from CIS511 at Penn. The course followed Michael Sipser's \textit{Introduction to the Theory of Computation (3ed)} text.}
	\vspace*{4mm}
}

%
% Convention for citations is authors' initials followed by the year.
% For example, to cite a paper by Leighton and Maggs you would type
% \cite{LM89}, and to cite a paper by Strassen you would type \cite{S69}.
% (To avoid bibliography problems, for now we redefine the \cite command.)
% Also commands that create a suitable format for the reference list.
\renewcommand{\cite}[1]{[#1]}
\def\beginrefs{\begin{list}%
		{[\arabic{equation}]}{\usecounter{equation}
			\setlength{\leftmargin}{2.0truecm}\setlength{\labelsep}{0.4truecm}%
			\setlength{\labelwidth}{1.6truecm}}}
	\def\endrefs{\end{list}}
\def\bibentry#1{\item[\hbox{[#1]}]}

%Use this command for a figure; it puts a figure in wherever you want it.
%usage: \fig{NUMBER}{SPACE-IN-INCHES}{CAPTION}
\newcommand{\fig}[3]{
	\vspace{#2}
	\begin{center}
		Figure \thelecnum.#1:~#3
	\end{center}
}
% Use these for theorems, lemmas, proofs, etc.
\newtheorem{theorem}{Theorem}[lecnum]
\newtheorem{lemma}[theorem]{Lemma}
\newtheorem{proposition}[theorem]{Proposition}
\newtheorem{claim}[theorem]{Claim}
\newtheorem{corollary}[theorem]{Corollary}
\newtheorem{definition}[theorem]{Definition}
\newenvironment{proof}{{\bf Proof:}}{\hfill\rule{2mm}{2mm}}

% **** IF YOU WANT TO DEFINE ADDITIONAL MACROS FOR YOURSELF, PUT THEM HERE:

\begin{document}
	%FILL IN THE RIGHT INFO.
	%\lecture{**LECTURE-NUMBER**}{**DATE**}{**LECTURER**}{**SCRIBE**}
	\chno{15}{PSPACE Completeness}{Sampath Kannan}{Zach Schutzman}



\section*{The Class $PSPACE\mhyphen Complete$}

\definition{A language $L$ is $PSPACE\mhyphen Complete$ if $L$ is in $PSPACE$ and for all $A\in PSPACE$, there exists a mapping reduction from $A$ to $L$ in polynomial time.}

\definition{A \textbf{quantified boolean formula} is a formula of the form $\forall x_1 \exists x_2 \forall x_3 \dots \phi(x_1,x_2,\dots)$, where $\phi$ is a boolean formula over the $x_i$.}

\claim{The language of True Quantified Boolean Formulas ($TQBF$) is $PSPACE\mhyphen Complete$.}

\begin{proof}
	
	We'll think of a machine that decides this language as taking a quantified boolean formula as input and accepting if it is true, rejecting if it is false.
	
	This language is obviously in $PSPACE$, we can simply check each allowable assignment and accept if the formula is true.
	
	To show that it is $PSPACE$-Complete, we need to show that every other $PSPACE$ language can be reduced to it in polynomial time.  Let $L$ be a language in $PSPACE$.  Without loss of generality, we can say that a decider for $L$ takes $O(2^{n^k})$ time. 
	
	We saw in the Cook-Levin Theorem how to think of states of computation as boolean formulas, and we will do something similar here.  Let $C_{init}$ the initial configuration and $C_{accept}$ as the canonical accepting configuration.  We write $\Phi(C_a,C_b,t)$ to denote the quantified boolean formula corresponding to `there exists a sequence of computation steps of length at most $t$ such that the machine moves from configuration $C_a$ to $C_b$.  We want to find formula equivalent to $\Phi(C_{init},C_{accept},2^{n^k})$ which is true if and only if the input $x$ to the decider for $L$ is in $L$.
	
	We start by observing that if $x\in L$, then there must be some intermediate configuration $C_m$ which the machine passes through in its computation.  We can then write \[\Phi(C_{init},C_{accept},2^{n^k}) \iff \Phi(C_{init},C_{m},\frac{2^{n^k}}{2})\land \Phi(C_{m},C_{accept},\frac{2^{n^k}}{2})\]
	
	We can also consider quarter-way and eighth-way points, and so on and do the same thing, however, we get an exponentially large formula by the time we get down to step sizes of $t=1$.
	
	We get around this by introducing two new variables $C_3,C_4$ and universally quantifying them over our configuration pairs $(C_1,m),(m,C_2)$ such that we have 

	\[ \Phi(C_1,C_2,t)=\exists m \forall(C_3,C_4)\in \{ (C_1,m),(m,C_2) \} (\Phi(C_3,C_4,\frac{t}{2})) \]
	
	We only need polynomial space for this, hence $L$ is reducible to $TQBF$, and $TQBF$ is $PSPACE$-Complete.
\end{proof}



\textbf{Example:} The language Generalized Geography ($GG$) is $PSPACE$-Complete.  $GG$ is a two player game on a directed graph.  One player picks a node, then the other picks a node for which there exists an arc from the previous player's choice into that node.  A player loses if there are no nodes for her to legally choose (i.e. the previous player picked a node with out degree zero).  Player One's first choice is a designated start node.  The language $GG$ is the decision problem of whether, given a directed graph with a start node, Player One can always win given perfect play.

We can think of this as a $TQBF$ in the following way:  Player One has a winning strategy if there exists a node with out-degree zero such that for any previous choice made by Player Two there exists a previous choice by Player One, and so on.

\end{document}