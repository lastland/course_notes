
%FILL IN THE RIGHT INFO.
%\lecture{**LECTURE-NUMBER**}{**DATE**}{**LECTURER**}{**SCRIBE**}
\chno{17}{Space Hierarchy}{Sampath Kannan}{Zach Schutzman}
%\footnotetext{These notes are partially based on those of Nigel Mansell.}

% **** YOUR NOTES GO HERE:

% Some general latex examples and examples making use of the
% macros follow.  
%**** IN GENERAL, BE BRIEF. LONG SCRIBE NOTES, NO MATTER HOW WELL WRITTEN,
%**** ARE NEVER READ BY ANYBODY.


\section*{Finishing up $NL=Co\mhyphen NL$}

\claim{$NL = Co\mhyphen NL$ (continued from last time)}

\begin{proof}
	
	(Continued)
	

	
	We were thinking about how to get the count of reachable vertices, $c$.  Let $c_i$ be the number of vertices reachable from $s$ by paths of length at most $i$.  We can, given $c_i$, compute $c_{i+1}$.  We have $c_0=1$, so if we have a procedure to compute the increments, we can inductively find $c_{n-1}$.  The idea will be to note that vertices reachable in at most $i+1$ steps have an edge from something reachable in at most $i$ steps.  
	
	Similarly, denote $s_i$ the set of vertices reachable from $s$ in $i$ or fewer steps.
	
	The following procedure will compute $c_{i+1}$: For each vertex $v_j$, $v_j\in s_{i+1}$ if $v_j\in s_i$ or there exists a vertex $u$ such that $u\in s_i$ and $(u,v_j)$ is an edge in $G$.  We will nondeterministically guess whether each vertex $v$ is in $s_{i}$.  To confirm it, we use a $NL$ procedure for $PATH$ limited to paths of length $i$.  We'll keep a counter for the number of things we find in $s_i$ and the number of things in $s_{i+1}$.  For each vertex, if we find it in $s_i$, then it is in $s_{i+1}$, so we increment both of our counters.  Else, if there is an edge from something in $s_i$ to it, then we increment our counter for $s_{i+1}$.  Otherwise, we move on to the next vertex.  After checking all vertices we now have the number of things in $s_{i+1}$ on the correct branch of the $PATH$ algorithm, which is the one that correctly computed the number of things in $s_i$.
	
	Inductively, we now know our $c$, as we do this procedure to find $c=c_{n-1}$.  Since we have $c$, we have an $NL$ algorithm to decide $\overline{PATH}$, hence $L=Co\mhyphen NL$.
	
	
	
	
	
\end{proof}


We have $L\subseteq NL = Co\mhyphen NL$.  Whether $L$ equals $NL$ is unknown.  The language $UPATH$, which is $PATH$ on undirected graphs is obviously in $NL$.  It was proven about 10 years ago that $UPATH$ is actually in $L$.

We can say that Savitch's Theorem still applies.  That is, $NSPACE(f(n)) \subseteq DSPACE(f(n)^2)$. Therefore, $NL\subseteq L^2$, deterministic log-squared space.


\section*{Space Hierarchy Theorem}

We now are ready to show our first proper separation of time and space classes.

\definition{A function $f(n)$ is `nice' if it is \textbf{fully space-constructible}. That is, there is a Turing machine which given input, say $1^n$, can compute $f(n)$ using no more than $f(n)$ space.}

Space-constructible functions include all the familiar ones, like monotonic polynomials, exponentials, logarithms, roots, etc.

\claim{(\textbf{The Space Hierarchy Theorem}) Let $f(n)$ be a `nice' function.  Then there is a language $L$ which can be decided by an $f(n)$-space deterministic Turing machine, but not by any deterministic Turing machine using $o(f(n))$ space.  That is, for $f(n)$, there is a language requiring at least $O(f(n))$ space to decide by a deterministic Turing machine.}

\begin{proof}
	
	We prove this by constructing such a language. via diagonalization.  Define a deterministic Turing machine $M$ which uses $f(n)$ space and $L(M)$ cannot be decided by any Turing machine using $o(f(n))$ space.  We'll think of $M$ as being different from every machine which uses $o(f(n))$ space.
	
	$M$ on input $\langle x\rangle $ first checks if $\langle x\rangle $ is a description of a Turing machine.  If not, reject.  If it is, $M$ should run $\langle x\rangle$ $x$.  If $x$ finishes in $f(n)$ space, then $M$ outputs the opposite answer as $x$.  $M$ stops and rejects if $\langle x \rangle$ exceeds $f(n)$ space.
	
	We also need to count the steps of $x$ to make sure we don't exceed $2^{O(f(n))}$ time.
	
	We also have an issue because $O(f(n))$ and $o(f(n))$ are asymptotic notions, so we may get something wrong for relatively small $n$.  
	
	We'll modify the behavior of $M$ slightly.  On any input, it checks to see if it is the description of a Turing machine, followed by $01^*$.  If not, reject.  If so, simulate $x$ on the whole input.  As before, if $x$ properly terminates, we output the complement of its answer.  Now, we have that if $x$ uses $g(n)$ space, we know there exists an $n_0$ such that for all $n>n_0$, $g(n)\leq f(n)$, because there is a sufficiently long input string such that $x$ requires no more space than $M$.
	
	
\end{proof}

We know $L\subsetneq PSPACE$, and by Savitch's Theorem, $NL\subsetneq PSPACE$.  We also have, if $r_1<r_2\in \mathbb{R}^+$, then $DSPACE(n^{r_1}) \subsetneq DSPACE(n^{r_2})$.

Similar results hold for $NSPACE$.

\section*{Time Hierarchy (briefly)}

We want to say that if $g(n)\in o(f(n))$, for time-constructible function $f(n)$, there is a language in $DTIME(f(n))$ that is not in $DTIME(g(n))$, but this isn't necessarily true.  If we try the same idea as for Space Hierarchy, we run $M$ on $x$ for $f(n)$ steps, but the problem is that we need to somehow count the steps, which requires keeping a counter of size $O(log(f(n)))$, which may all need to be altered after a step.  Additionally, we need access to the transition function of $x$, which we can keep in space, but in time, we need to track back and forth for lookups, which again adds a quadratic blowup.  We can keep this description along with us, but the counter is still a problem to update.  

We therefore need to rephrase the statement of the theorem to be

\claim{If there is a language in $g(n) \in o(f(n))/log(f(n))$, then there is a language in $DTIME(f(n))$ not in $DTIME(g(n))$.}

The proof follows the same structure as the Space Hierarchy Theorem, including $log(f(n))$ additional steps to update the counter.



